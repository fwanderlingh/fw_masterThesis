\section{PatrolGRAPH*}

The logic behind the PatrolGRAPH* follows a different approach. First of all, although it is presented as online algorithm, it also presents an offline phase which is needed in order to optimise the successive online phase. So we can classify it in the middle of the two approaches. It takes into account not only information about the vertex count, but also on the edge traversal count. In particular the decision that is made in vertex $v_i$ to choose the next vertex $v_j$ is based on the probability associated to the edge that connects them $e(i,j)$.
The algorithm is designed to solve the problem of Multi--Robot Controlled Frequency Coverage, in which a team of robots are requested to repeatedly visit a set of pre--defined locations of the environment according to a specified \emph{frequency distribution}. In particular we want the frequency distribution to be \emph{uniform} in order to visit all the vertices of the graph in the shortest time, and this kind of formulation is known as the Multi--Robot Uniform Frequency Coverage (MRUFC). In particular the MRUFC problem corresponds to the case when ${\bm{\lambda}}^*$ has a uniform distribution, i.e., for all $s_i$, $\lambda_i^* = 1/N$.  The frequency distribution of visits in particular is controlled by tuning the so called \emph{transition matrix} $P$, which contains for each edge $e_{ij}$ the probability $p_{ij}$ of being traversed. $P$ is subject to the following constraints:

\begin{equation}
\sum\limits_{j=1}^{N} {p_{ij} } = 1, \quad (i=1,\ldots,N),
\label{Pmatrix3b}
\end{equation}

\begin{equation}
0 \leq p_{ij} \leq 1,\quad (i,j=1,\ldots,N). 
\label{Pmatrix3c}
\end{equation}
 
In its simplest form the elements of $P$ will be:

\begin{equation}
p_{ij}=\frac{1}{|E_i|},
\label{eq_adj}
\end{equation}
if $v_i$ and $v_j$ are adjacent, while if $s_i$ and $s_j$ are not adjacent,
\begin{equation}
p_{ij}=0.
\label{eq_noadj}
\end{equation}

It is easy to understand that with this kind of formulation every vertex will receive a number of visits which is proportional to its incoming edges, and we want instead all the vertices to be visited uniformly. To accomplish this the PatrolGRAPH* performs the so called \emph{offline phase} to tune the $P$ transition matrix.

%%%%%%%%%%%%%%%%%%%%%%%%%%%%%%
\subsection{The offline phase}

Let $\bm{1}=[1 \ldots 1]^T$ be the unitary vector. The flow balance equations in matrix form, by considering the additional requirement that the sum of average visiting rates, computed over all vertices of $\hat{G}_N$, is constant, can be written as:

\begin{equation}
\left[\begin{array}{*{20}c} {P^T - I} \\ {\bm 1^T} \end{array} \right] \bm \lambda = \left[\begin{array}{*{20}c} {0} \\ \vdots \\ {0} \\ {C} \end{array}\right].
\label{theorem45}
\end{equation}

The system \eqref{theorem45} has exactly one solution whenever $P$ is stochastic and irreducible \cite{5711675}. The last equation in \eqref{theorem45} allows one to determine the unique solution whose components sum up to $C$.

The off--line phase then has the main purpose of finding a solution to the so--called \emph{inverse problem}, i.e., the problem of choosing $P$ such as to guarantee that the solution of \eqref{theorem45}, with $C=1$, corresponds to the prescribed frequency distribution. The off--line phase does not involve the individual robots.

Let ${\bf{b}} = [b_{1}\cdots b_{N^2}]^T\in \Re^{N^2}$ be defined as a vector which parametrize the elements of the transition matrix $P$, so that $P=P({\bf{b}})\in\Re^{N\times N}$. This is done by ideally subdividing $\bf{b}$ into $N$ vectors with length $N$, and by stacking them onto each other to build up the rows of $P$. More formally, 
\begin{equation}
p_{ij}=b_{N \times (i - 1) + j}, \quad (i,j=1,\ldots,N),
\label{ths0}
\end{equation}

and $\bm{\lambda^*}$ represent the ideal desired solution of the MRCFC problem. Then, ${\bm{\lambda^*} - \bm{\lambda}({\bf{b}})}$ is a representation of the mismatch between the actual frequency distribution and the desired one as a function of the structure of the navigation graph (expressed by the entries of matrix $P$). 

The off--line phase of the \textit{PatrolGRAPH*} algorithm is the search for a solution to the following minimization problem with respect to the unknown variable ${\bf{b}}$.


\begin{equation}
{\bf{b^*}} = \arg \min_{{\bf{b}}}({\bm{\lambda^*} - \bm{\lambda}({\bf{b}})})^T ({\bm{\lambda^*} - \bm{\lambda}({\bf{b}})}),
\label{ths3}
\end{equation}
\indent subject to
\begin{equation}
\sum\limits_{j=1}^{N} {b_{N\times(i-1)+j} } = 1, \quad (i=1,\ldots,N), \\
\label{con3a}
\end{equation}
\begin{equation}
b_{N\times(i-1)+j} \ge  |\epsilon |,\quad (\forall i,j\quad s.t.\quad a_{ij} \in A_i),  
\label{con3b}
\end{equation}
\begin{equation}
b_{N\times(i-1)+j} = 0,\quad (\forall i,j\quad s.t.\quad a_{ij} \notin A_i). 
\label{con3c}
\end{equation}


%%%%%%%%%%%%%%%%%%%%%%%%%%%%%%
\subsection{The online phase}
After performing the previous steps, the online phase follows the same pattern of the previous ones as in Algorithm \ref{alg:rt_gen}, but now the choose operator is more elaborated:

\begin{algorithm}
\begin{algorithmic}[1]
\STATE{$c(v) = c(v) + 1$} \label{a:PG_upd_cv}
\FORALL{$j$ such that $e_{cj}\in E_c$} %\label{a:initU_rt}
	\STATE{$\Delta p_{cj} = (k_{cj}- \eta_{j})/v_c - p_{cj}$}\label{a:PG_deltaP}
\ENDFOR
\STATE{$l = arg min_{j}( \Delta p_{cj})$}\label{a:PG_find_min}
\STATE{$k_{cl} = k_{cl} + 1$}\label{a:PG_upd_k}
\RETURN{$e_{cl}$}
\end{algorithmic}
\caption{Choose operator for PatrolGRAPH*}\label{alg:rt_pg}
\end{algorithm}

Where: $c(v)$ same as before is the vertex count. $k_{ij}$ is the edge count. $\eta_j$ is a zero mean random variable with standard deviation $\sigma$, that is, $\eta_j=N(0,\sigma)$. In particular $\eta$ is used to purposely introduce a factor of unpredictability but since we don't need this feature we will simply set it to zero. Line \ref{a:PG_upd_cv} updates the number of visits $c(v)$ received by $v_c$; Line \ref{a:PG_deltaP} computes, for every adjacent vertex, the error $\Delta p_{cj}$ between the ratio $k_{cj}/v_c$ and the desired relative frequency $p_{cj}$; Line \ref{a:PG_find_min} picks the edge $a_{cl}$ for which $\Delta p_{cj}$ is minimum; Line \ref{a:PG_upd_k} updates $k_{cl}$. 
In \cite{5711675} is demonstrated that the routing policy adopted by \textit{PatrolGRAPH*} in Algorithm \ref{alg:rt_pg} to distribute robots along edges $a_{ij}$ departing from $s_i$ converges to the desired relative frequencies $p_{ij}$.

To demonstrate the effectiveness of the offline phase of the PatrolGRAPH* the algorithm is tested on a 5x5 grid obstacle-free map (cell side size 5 m) using 3 quadcopters, running four simulations with two different time horizons: 

\begin{itemize}
\item 10 minutes patrolling without and with (figures \ref{fig:PG_pic1} and \ref{fig:PG_pic2})
\item 20 minutes patrolling without and with (figures \ref{fig:PG_pic3} and \ref{fig:PG_pic4})
\end{itemize}

The figures mentioned in the list are a graphical representation of the visit count for the cells in the map. In particular the visit count is encoded as a colour and the colour-bar is the reference. First of all it is important to know that the starting point for all the quadcopters is the cell in position (2,0), and this of course affects the propagation of visit in the neighbouring cells. In Figure \ref{fig:PG_pic1} and \ref{fig:PG_pic3} we can observe how without running the offline phase the corner vertices, which have the lowest number of in-going edges, receive a significantly smaller amount of visits (colour near blue).  In Figure \ref{fig:PG_pic2} and \ref{fig:PG_pic4}, instead, we notice how the visit distribution is much more uniform, and as the simulation time increases the coverage becomes more uniform.
\begin{figure}[h] 
  \label{fig:PG_countMaps}
  \centering
  \begin{minipage}[b]{0.45\linewidth}
    \centering
    \input{img/tikzPlots/5x5_unoptim_10min.tikz}
    \caption{Unoptimised PG* 10 min}
    \label{fig:PG_pic1}
    \vspace{4ex}
  \end{minipage}
  \quad
  \begin{minipage}[b]{0.45\linewidth}
    \centering
    % This file was created by matlab2tikz v0.4.7 running on MATLAB 8.0.
% Copyright (c) 2008--2014, Nico Schlömer <nico.schloemer@gmail.com>
% All rights reserved.
% Minimal pgfplots version: 1.3
% 
% The latest updates can be retrieved from
%   http://www.mathworks.com/matlabcentral/fileexchange/22022-matlab2tikz
% where you can also make suggestions and rate matlab2tikz.
% 
\begin{tikzpicture}

\begin{axis}[%
width=.75\linewidth,
%height=3.565625in,
axis on top,
%scale only axis,
xtick={0,1,...,4},
ytick={0,1,...,4},
xmin=-0.5,
xmax=4.5,
y dir=reverse,
ymin=-0.5,
ymax=4.5,
colormap/jet,
colorbar,
point meta min=0,
point meta max=18
]
\addplot [forget plot] graphics [xmin=-0.5,xmax=4.5,ymin=-0.5,ymax=4.5] {img/tikzPlots/5x5_optim_10min.png};
\end{axis}
\end{tikzpicture}%
    \caption{Optimised  PG* 10 min}
    \label{fig:PG_pic2}
    \vspace{4ex}%%
  \end{minipage}
  \begin{minipage}[b]{0.45\linewidth}
    \centering
    \input{img/tikzPlots/5x5_unoptim_20min.tikz}
    \caption{Unoptimised  PG* 20 min} 
    \label{fig:PG_pic3}
    \vspace{4ex}
  \end{minipage}
  \quad
  \begin{minipage}[b]{0.45\linewidth}
    \centering
    \input{img/tikzPlots/5x5_optim_20min.tikz}
    \caption{Optimised  PG* 20 min}
    \label{fig:PG_pic4}
    \vspace{4ex}%% 
  \end{minipage} 
\end{figure}