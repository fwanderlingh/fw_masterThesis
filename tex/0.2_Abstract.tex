\begin{center}
\Large {\textbf{Swarm of autonomous MAVs in Urban Search and Rescue applications}}\\
\large {by  \\
Francesco Wanderlingh}
\end{center}

\vskip 1cm
\Large {\textbf{Abstract}} \\ 

\large{
In the framework of swarm robotics involved in search and rescue applications, the main purpose of this thesis is to investigate complete 
terrain coverage algorithms. These kinds of algorithms are intended in the first place to perform a fast monitoring by means of aerial 
imaging, but many other applications are possible. Two main classes can be identified in the algorithms developed for the terrain coverage: 
online and offline coverage methods. For the online class the algorithms taken into account are Node Counting, Learning Real-Time A*, Edge 
Counting and PatrolGRAPH*. For the offline class instead the coverage problem was formulated as a Vehicle Routing Problem and solved with a 
greedy approach. In a first phase the algorithms were tested in a 3D simulation environment and a large number of results was collected and 
compared. Finally the whole framework was tested performing the coverage using real multi-copters. }
\clearpage
\null\newpage

\begin{center}
\Large {\bf{Zastosowania roju autonomicznych miniaturowych statków powietrznych w poszukiwaniu i ratownictwie miejskim}}\\
\large {przez  \\
Francesco Wanderlingh}
\end{center}

\vskip 1cm
\Large {\textbf{Streszczenie}} \\


\large{
Głównym celem pracy jest analiza algorytmów całkowitego pokrycia terenu z~użyciem roju miniaturowych statków powietrznych 
w~poszukiwaniu i ratownictwie miejskim. Algorytmy te są przeznaczone, przede wszystkim, do szybkiego monitorowania terenu 
z~użyciem obrazowania z~powietrza, aczkolwiek możliwych jest wiele innych zastosowań. Wyróżnia się dwie główne klasy algorytmów 
przeszukania terenu: metody \emph{on-line} i \emph{off-line}. Do pierwszej z nich można zaliczyć algorytmy: Node Counting, 
Learning Real-Time A*, Edge Counting and PatrolGRAPH*. W~podejściu \emph{off-line} zadanie pokrycia terenu zostało sformułowane 
jako problem marszrutyzacji i rozwiązano go za pomocą algorytmów zachłannych przeszukiwania grafu. W pierwszej fazie badań algorytmy 
były testowane w~symulacji w~stworzonym środowisku 3D. Wykonano wiele testów, a ich wyniki zostały zebrane i porównane. Następnie przeprowadzono 
eksperymenty pokrycia terenu z~użyciem rzeczywistych multikopterów.   
}
\clearpage
