% Please add the following required packages to your document preamble:
% \usepackage{booktabs}
% \usepackage{multirow}
\begin{table}[H]
\centering
\begin{tabular}{@{}cccccccc@{}}
\toprule
Set                & \multicolumn{4}{c}{Parameters}  & \multicolumn{3}{c}{Results}              \\ \cmidrule(lr){2-5} \cmidrule(lr){6-8}
                   & $\upomega$ & \#Rob & Grid Size & $\min c$ & $\max(\ell_i)$ & $\sum(\ell_i)$ & $\sigma(\ell_i)$ \\ \midrule
\multirow{3}{*}{1} & 1           & 3     & 4x4       &   1     &  25       &  74       &   0.662   \\
                             & 1           & 3     & 6x6       &   1     &   65      &  192      &   0.908    \\
                             & 1           & 3     & 8x8       &   1     &   150     &  447     &   1.17    \\ \midrule
\multirow{3}{*}{2} & 1           & 3     & 6x6       &   1     &   61      &  182      &  0.836     \\
                             & 2           & 3     & 6x6       &   1     &   66      &  193      &  1.19     \\
                             & 3           & 3     & 6x6       &   1     &   69      &  204      &  1.39     \\ \midrule
\multirow{3}{*}{3} & 1           & 2     & 6x6       &   1     &  88       &  174      &  0.832        \\
                             & 1           & 4     & 6x6       &   1     &  51       &  201      &  0.852        \\
                             & 1           & 6     & 6x6       &   1     &  41       & 241       &  0.539           \\ \midrule
\multirow{4}{*}{4} & 1           & 3     & 6x6       &   1     &  180     &   533     &    3.366        \\
                             & 1           & 3     & 6x6       &   2     &  114     &  339      &   0.816        \\
                             & 1           & 3     & 6x6       &   5     &  70      &   207      &   0.816       \\
                             & 1           & 3     & 6x6       &   10   &  66      &    198     &  0        \\ \bottomrule
\end{tabular}
\caption{Results of the simulation for the PatrolGRAPH* Alg.}
\label{tab:PG_results}
\end{table}

