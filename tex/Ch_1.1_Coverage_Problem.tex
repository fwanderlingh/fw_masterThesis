\section{The Coverage Problem}

This thesis work deals with the problem of finding the most effective algorithm to solve the coverage problem. Let's then start defining what the coverage problem deals with and how we can solve it.

 The are two major classes in which we can divide coverage problems: \emph{static} coverage and \emph{dynamic} coverage. Given a finite area with obstacles to cover, the static coverage solves the problem of finding the best robot positions such that visibility of the area is maximized. In this case then there will be some which will not be monitored by the robots (blind spots) and for our SAR purposes this is unacceptable. For this reason in the thesis we opted for a dynamic approach to coverage. In a dynamic coverage the robot will have to cover the area by exploring it thoroughly, so even if at each time step we just observe a little portion of the terrain, the final knowledge of the area is complete. If for example we have to find a victim in a stricken environment and the person is hidden in a blind spot, the static coverage does not ensures that the victim will be detected, while the dynamic coverage does. A static coverage moreover does not observe the area uniformly; the regions near to the robot will be well monitored while far one will be observed only marginally.
 
Now in the the field of dynamic coverage we can identify two classes:
\begin{itemize}
\item \textbf{Offline coverage}: the paths are pre-calculated with a routing algorithm, before the robot start exploring the environment
\item \textbf{Online coverage}: the robots decide their path while they are exploring the environment, by making decisions based on the current knowledge they have on the coverage completion
\end{itemize}

An important aspect to take into account is whether the coverage is meant to be performed by a single robot, or by a group of robots. The main difference between the two is that in the single robot the robot does not have to interact with other moving units and this simplifies the planning and does not require a inter-robot communication system. In the multi-robot coverage instead the communication between units is essential to take advantage of a collaborative behaviour. In this thesis the main objective has been to build a flexible software framework that can  be easily extend from a single-robot to multi-robot configuration seamlessly.

For known environments it is important to accomplish the coverage task in the minimum possible time and using this a priori knowledge we can construct an algorithm that computes the optimal paths to completely cover the area. This is possible decomposing the environment in a finite number of spots to visit (see next section, \ref{sec:spaceDec}) and by modelling the coverage as a routing problem where we have to visit all the location at least once. But due to the huge size of the solution space, using offline algorithms to find optimal path turns out to be not so trivial, optimal solution are really hard to compute, actually NP-hard \cite{wiki:VRP}, so in practice heuristic and deterministic methods have been developed that find acceptably good solutions for the VRP. This involves a discrete computation capability and in the perspective of using the algorithms to guide autonomous robots this can represent a limitation. That is why several online strategies have also been investigated.
From a computational point of view online algorithms are far less challenging. In an online coverage basically, every time the robot reaches a certain position on the map, it has the only simple task of choosing what point to reach next basing its decision only the local knowledge it has on the map (\emph{local} with respect to its position in space). It also has the advantage of being more flexible; think of a situation where a robot fails due to a crash: in an offline coverage the path are pre-calculated so the path assigned to that robot will not be covered, in an online case instead the robots will continue exploring the area till they know that the whole area is covered. They also fit better the swarm robotics approach where complex behaviours emerge from the sum of many simple ones.

As a final remark since performing SLAM (simultaneous localization and mapping) is out of the scope of this thesis the terrain is assumed to be known. Having in mind that by using any modern web mapping service the topology of the area can be retrieved this can be a reasonable assumption in real world applications.

The path planning required to perform the complete coverage task must be build on top of an abstract representation of the terrain. This abstract representation can be obtained through one of the various methods discussed in the next section.

